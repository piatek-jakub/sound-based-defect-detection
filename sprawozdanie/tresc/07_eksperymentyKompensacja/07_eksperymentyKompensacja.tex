\section{Eksperymenty z kompensacją klas niezbalansowanych}
Niniejszy rozdział poświęcony jest próbie poprawy jakości klasyfikacji wyrażonej za pomocą miar \textit{precision}, \textit{recall} oraz \textit{F1-score} dla atrybutów, które we wcześniejszych eksperymentach okazały się najtrudniejsze do poprawnej predykcji, tj. \textit{Gears}, \textit{Belt} oraz \textit{Curved railway track}. Atrybuty te charakteryzują się zarówno znacznym niezbalansowaniem klas, jak i obecnością wielu klas anomalnych o zbliżonych właściwościach akustycznych.

W ramach przeprowadzonych eksperymentów zastosowano metody kompensacji niezbalansowania klas na etapie uczenia modelu, bez ingerencji w oryginalny rozkład danych wejściowych. W szczególności wykorzystano mechanizm ważenia klas (\textit{class weighting}), który zwiększa wpływ próbek klas mniejszościowych na proces optymalizacji modelu. Celem takiego podejścia było zwiększenie czułości modelu na klasy rzadkie oraz poprawa wartości miar \textit{recall} i \textit{F1-score}, które są kluczowe z punktu widzenia detekcji stanów uszkodzeń.

W kolejnych podrozdziałach przedstawiono szczegółowe wyniki eksperymentów oraz analizę wpływu zastosowanych metod na skuteczność klasyfikacji dla wybranych atrybutów.

\subsection{Zastosowane metody kompensacji klas niezbalansowanych}
W ramach niniejszej sekcji wprowadzono szereg modyfikacji mających na celu ograniczenie negatywnego wpływu niezbalansowania klas na skuteczność klasyfikacji:
\begin{itemize}
    \item \textbf{Ważenie klas w procesie uczenia} - podstawową metodą kompensacji niezbalansowania klas było zastosowanie mechanizmu ważenia klas (\textit{class weighting}) w modelu regresji logistycznej. Wykorzystano parametr \texttt{class\_weight='balanced'}, który powoduje automatyczne obliczanie wag klas w sposób odwrotnie proporcjonalny do liczby próbek danej klasy. W rezultacie błędy popełniane na klasach mniejszościowych miały większy wpływ na wartość funkcji kosztu, co sprzyjało poprawie czułości modelu względem rzadkich stanów uszkodzeń.
    \item \textbf{Rozszerzenie przestrzeni cech akustycznych} - wstępne eksperymenty wykazały ograniczoną skuteczność samego ważenia klas, dlatego wprowadzono dodatkowe ulepszenia polegające na rozszerzeniu zbioru cech audio. Zwiększono liczbę współczynników MFCC do 40 oraz zastosowano większe okno transformaty FFT. Ponadto uwzględniono dodatkowe cechy spektralne, takie jak \textit{spectral centroid}, \textit{spectral rolloff}, \textit{zero crossing rate}, cechy chromatyczne oraz reprezentację harmoniczną \textit{Tonnetz}. W efekcie liczba cech opisujących pojedynczą próbkę wzrosła ponad dwukrotnie, co umożliwiło lepsze odwzorowanie subtelnych różnic pomiędzy poszczególnymi stanami urządzenia.
    \item \textbf{Dostrojenie parametrów modelu i progów decyzyjnych} -dostrojono parametry modelu regresji logistycznej, w tym współczynnika regularyzacji oraz maksymalnej liczby iteracji algorytmu optymalizacyjnego. Dodatkowo zastosowano procedurę dostrajania progów decyzyjnych, której celem było uzyskanie lepszej równowagi pomiędzy precyzją a czułością dla klas mniejszościowych. Podejście to umożliwiło zwiększenie skuteczności wykrywania rzadkich defektów przy jednoczesnym ograniczeniu liczby fałszywych alarmów.
\end{itemize}

Dodatkowo, utworzono trzy modele z zróżnicowaną zdolnością klasyfikacji:
\begin{itemize}
    \item Model wieloklasowy wielokryterialny - model klasyfikujący w ten sam sposób, co model z wstępnego eksperymentu, lecz z zastosowanymi metodami kompensacji klas niezbalansowanych - każda predykcja opiera się na określeniu stanu każdego z elementów, gdzie każdy element potrafi mieć nawet kilka możliwych anomalii,
    \item Model wieloklasowy binarny - model, w którym każda predykcja opiera się na określeniu stanu każdego z elementów urządzenia mechanicznego, ale w sposób binarny (Dobry/Wadliwy),
    \item Model jednoklasowy binarny - model opiera swoją predykcję o binarną analizę systemu jako całości (działa poprawnie lub jest awaria gdziekolwiek w systemie). Z uwagi na charakter modelu, ocena poprawności działania modelu opierać się będzie na całych zbiorach, a nie jedynie na 3 klasach wybranych do eksperymentów w przedstawionej sekcji.
\end{itemize}

Zastosowane modyfikacje miały na celu poprawę jakości klasyfikacji w szczególności dla klas rzadkich, co powinno przejawiać się wzrostem wartości \textit{recall} oraz \textit{Macro F1-score}, zmniejszeniem różnicy pomiędzy średnimi typu \textit{macro} i \textit{weighted}, a także ogólną poprawą zdolności modelu do detekcji rzadko występujących stanów uszkodzeń.
\subsection{Wyniki eksperymentów po zmianach kompensacyjnych - model wieloklasowy wielokryterialny}
\subsubsection{Gears (Przekładnie)}
\begin{figure}[H]
    \centering
    \includegraphics[width=0.7\textwidth]{reports_after_adjust/multiclass_multiclassifier/report_ToyCar_Gears.png}
    \caption{Raport klasyfikacji atrybutu Gears dla zbioru ToyCar}
    \label{fig:toycar_gears}
\end{figure}

\subsubsection{Belt (Pas)}
\begin{figure}[H]
    \centering
    \includegraphics[width=0.7\textwidth]{reports_after_adjust/multiclass_multiclassifier/report_ToyConveyor_Belt.png}
    \caption{Raport klasyfikacji atrybutu Belt dla zbioru ToyConveyor}
    \label{fig:toyconveyor_belt}
\end{figure}

\subsubsection{Curved railway track (Zakrzywiony tor)}
\begin{figure}[H]
    \centering
    \includegraphics[width=0.7\textwidth]{reports_after_adjust/multiclass_multiclassifier/report_ToyTrain_Curved railway track.png}
    \caption{Raport klasyfikacji atrybutu Curved railway track dla zbioru ToyTrain}
    \label{fig:toytrain_curvedtrack}
\end{figure}


\subsection{Wyniki eksperymentów po zmianach kompensacyjnych - model wieloklasowy binarny}
\subsubsection{Gears (Przekładnie)}
\begin{figure}[H]
    \centering
    \includegraphics[width=0.7\textwidth]{reports_after_adjust/multiclass_binary/report_ToyCar_Gears.png}
    \caption{Raport klasyfikacji atrybutu Gears dla zbioru ToyCar}
    \label{fig:toycar_gears}
\end{figure}

\subsubsection{Belt (Pas)}
\begin{figure}[H]
    \centering
    \includegraphics[width=0.7\textwidth]{reports_after_adjust/multiclass_binary/report_ToyConveyor_Belt.png}
    \caption{Raport klasyfikacji atrybutu Belt dla zbioru ToyConveyor}
    \label{fig:toyconveyor_belt}
\end{figure}

\subsubsection{Curved railway track (Zakrzywiony tor)}
\begin{figure}[H]
    \centering
    \includegraphics[width=0.7\textwidth]{reports_after_adjust/multiclass_binary/report_ToyTrain_Curved railway track.png}
    \caption{Raport klasyfikacji atrybutu Curved railway track dla zbioru ToyTrain}
    \label{fig:toytrain_curvedtrack}
\end{figure}



\subsection{Wyniki eksperymentów po zmianach kompensacyjnych - model jednoklasowy binarny}
\subsubsection{ToyCar}
\begin{figure}[H]
    \centering
    \includegraphics[width=0.7\textwidth]{reports_after_adjust/singleclass_binary/report_ToyCar.png}
    \caption{Raport klasyfikacji zbioru ToyCar}
    \label{fig:toycar}
\end{figure}

\subsubsection{ToyConveyor}
\begin{figure}[H]
    \centering
    \includegraphics[width=0.7\textwidth]{reports_after_adjust/singleclass_binary/report_ToyConveyor.png}
    \caption{Raport klasyfikacji zbioru ToyConveyor}
    \label{fig:toyconveyor}
\end{figure}

\subsubsection{ToyTrain}
\begin{figure}[H]
    \centering
    \includegraphics[width=0.7\textwidth]{reports_after_adjust/singleclass_binary/report_ToyTrain.png}
    \caption{Raport klasyfikacji zbioru ToyTrain}
    \label{fig:toytrain}
\end{figure}

\subsection{Podsumowanie i analiza wyników po kompensacji klas niezbalansowanych}
\subsubsection{Model wieloklasowy wielokryterialny}
\begin{itemize}
    \item Zastosowane metody kompensacji klas niezbalansowanych doprowadziły do istotnej poprawy jakości klasyfikacji we wszystkich analizowanych zbiorach danych. Największe korzyści zaobserwowano dla klas mniejszościowych reprezentujących różne rodzaje anomalii, które przed wprowadzeniem modyfikacji charakteryzowały się niską czułością oraz niestabilnymi wynikami.
    \item W przypadku zbioru \textit{Gears} odnotowano wyraźny wzrost wartości miary \textit{recall} dla klas \textit{Deformed} oraz \textit{Melted} (odpowiednio z 0.57 do 0.91 oraz z 0.49 do 0.92), co świadczy o znacznie lepszej zdolności modelu do wykrywania rzadkich stanów uszkodzeń. Jednocześnie utrzymano bardzo wysoką skuteczność klasyfikacji klasy \textit{Normal}, co potwierdza brak negatywnego wpływu wprowadzonych zmian na klasę dominującą.
    \item Analogiczny trend zaobserwowano dla zbioru \textit{Belt}, gdzie nastąpiła istotna poprawa czułości dla klas anormalnych \textit{Attached metallic object 2} oraz \textit{Attached metallic object 3}. Wartości miary \textit{recall} wzrosły odpowiednio z 0.76 do 0.94 oraz z 0.56 do 0.92, co przełożyło się na znaczący wzrost miary F1 oraz średniej makro, szczególnie istotnej w kontekście danych niezbalansowanych.
    \item Dla zbioru \textit{Curved railway track} poprawa jakości klasyfikacji objęła wszystkie klasy anomalii, czego potwierdzeniem jest wzrost miary F1 dla klas \textit{Broken}, \textit{Disjointed} oraz \textit{Obstructing stone}. Model po modyfikacjach osiągnął bardzo wysoką wartość dokładności ogólnej (0.98) oraz wyraźnie lepszą średnią makro w porównaniu do wyników sprzed kompensacji.
    \item We wszystkich przypadkach zaobserwowano istotny wzrost wartości średniej makro miary F1, co wskazuje na bardziej zrównoważone traktowanie wszystkich klas przez model. Potwierdza to skuteczność zastosowanych technik kompensacji niezbalansowania klas, w szczególności ważenia klas oraz rozszerzenia przestrzeni cech akustycznych.
\end{itemize}

Uzyskane wyniki pokazują, że mimo zachowania wieloklasowego charakteru problemu, odpowiednie dostosowanie procesu uczenia pozwala znacząco poprawić wykrywalność rzadkich anomalii bez istotnego pogorszenia jakości klasyfikacji stanów normalnych.

\subsubsection{Model wieloklasowy binarny}
\begin{itemize}
    \item Dla zbioru \textit{Gears} model binarny osiągnął bardzo wysoką czułość klasy \textit{Defect} (0.96), co oznacza znaczącą poprawę zdolności wykrywania uszkodzeń w porównaniu do modelu wieloklasowego. Jednocześnie utrzymano wysoką precyzję i czułość dla klasy \textit{Normal}, co przełożyło się na wzrost dokładności ogólnej do poziomu 0.97.
    \item Dla zbioru \textit{Belt} przejście na klasyfikację binarną przyniosło jedynie nieznaczną poprawę wyników w porównaniu do modelu wieloklasowego. Osiągnięto bardzo wysoką czułość klasy \textit{Defect} (0.99), jednak kosztem obniżenia precyzji, co sugeruje większą liczbę fałszywych alarmów w tym wariancie modelu.
    \item W przypadku zbioru \textit{Curved railway track} zaobserwowano największą poprawę jakości klasyfikacji. Model binarny osiągnął niemal idealne wyniki dla obu klas, z miarą F1 równą 0.97 dla klasy \textit{Defect} oraz 0.99 dla klasy \textit{Normal}. Wskazuje to, że w tym scenariuszu detekcja samego faktu wystąpienia anomalii jest znacznie łatwiejsza niż jej szczegółowa kategoryzacja.
\end{itemize}
Porównanie wyników modeli wieloklasowych i binarnych wskazuje, że scalenie klas anomalii prowadzi do dalszej poprawy skuteczności detekcji defektów, szczególnie w zbiorach o silnym niezbalansowaniu i relatywnie jednorodnych cechach akustycznych anomalii. Jednocześnie podejście to eliminuje możliwość rozróżniania poszczególnych typów uszkodzeń, co stanowi istotne ograniczenie z punktu widzenia diagnostyki przyczynowej.

\subsubsection{Model jednoklasowy binarny}
\begin{itemize}
    \item Dla wszystkich analizowanych zbiorów danych (\textit{ToyCar}, \textit{ToyConveyor} oraz \textit{ToyTrain}) model osiągnął niemal idealne wyniki klasyfikacji. Wartości miar \textit{precision}, \textit{recall} oraz \textit{F1-score} dla obu klas wyniosły w większości przypadków 1.00, a dokładność ogólna osiągnęła poziom 1.00.
    \item Uzyskane rezultaty wskazują, że przy tak dużej liczbie próbek oraz wyraźnej separowalności cech akustycznych pomiędzy stanami normalnymi i anormalnymi, nawet przy wydzieleniu 20\% danych do zbioru testowego model jest w stanie niemal bezbłędnie wykrywać obecność anomalii w systemie.
    \item Należy jednak podkreślić, że tak wysoka skuteczność wynika w dużej mierze z uproszczenia problemu klasyfikacji oraz globalnego charakteru etykiet. Model nie rozróżnia typów ani źródeł anomalii, a jedynie sygnalizuje fakt wystąpienia nieprawidłowości gdziekolwiek w systemie.
    \item W konsekwencji, omawiany wariant modelu może być szczególnie przydatny w systemach monitoringu i wczesnego ostrzegania, gdzie kluczowe znaczenie ma szybkie i niezawodne wykrycie stanu awaryjnego. Jednocześnie jego zastosowanie w zaawansowanej diagnostyce technicznej jest ograniczone ze względu na brak informacji o charakterze oraz lokalizacji defektu.
\end{itemize}

