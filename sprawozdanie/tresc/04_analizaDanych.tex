\section{Analiza zbioru danych}

\begin{table}[htbp]
\centering
\begin{tabular}{|l|l|l|l|}
\hline
Atrybut & Stan     & Ilość próbek (nagrań) & Długość nagrań {[}h{]} \\ \hline
Gears   & Deformed & 1436                  & 4.39                   \\ \hline
Gears   & Melted   & 1440                  & 4.40                   \\ \hline
Gears   & Normal   & 25644                  & 517.49                 \\ \hline
Shaft   & Bent   & 2160                  & 6.60                 \\ \hline
Shaft   & Normal   & 26360                  & 519.68                 \\ \hline
Tires   & Coiled (plastic ribbon)   & 1436                  & 4.39                 \\ \hline
Tires   & Coiled (steel ribbon)   & 1440                  & 4.40                 \\ \hline
Tires   & Normal   & 25644                  & 517.49                 \\ \hline
Voltage   & Normal   & 25640                  & 517.48                 \\ \hline
Voltage   & Over voltage   & 1440                  & 4.40                 \\ \hline
Voltage   & Under voltage   & 1440                  & 4.40                 \\ \hline
\end{tabular}
\caption{Tabela nagrań dla zbioru ToyCar}
\end{table}

\begin{table}[htbp]
\centering
\begin{tabular}{|l|l|l|l|}
\hline
Atrybut & Stan     & Ilość próbek (nagrań) & Długość nagrań {[}h{]} \\ \hline
Belt   & Attached metallic object 1 & 720                  & 2.00                   \\ \hline
Belt   & Attached metallic object 2   & 1260                  & 3.50                   \\ \hline
Belt   & Attached metallic object 3   & 1260                  & 3.50                 \\ \hline
Belt   & Normal   & 23492                  & 182.60                 \\ \hline
Belt   & Removed   & 20                  & 0.06                 \\ \hline
Tail pulley   & Excessive Tension   & 1500                  & 4.17                 \\ \hline
Tail pulley   & Normal   & 23892                  & 183.71                 \\ \hline
Tail pulley   & Removed   & 1360                  & 3.78                 \\ \hline
Tension pulley   & Aging   & 540                  & 1.50                 \\ \hline
Tension pulley   & Excessive Tension   & 1980                  & 5.50                 \\ \hline
Tension pulley   & Normal   & 24232                  & 184.66                 \\ \hline
Voltage   & Normal   & 23772                  & 183.38                 \\ \hline
Voltage   & Over voltage   & 1480                  & 4.11                 \\ \hline
Voltage   & Under voltage   & 1500                  & 4.17                 \\ \hline
\end{tabular}
\caption{Tabela nagrań dla zbioru ToyConveyor}
\end{table}

\begin{table}[htbp]
\centering
\begin{tabular}{|l|l|l|l|}
\hline
Atrybut & Stan     & Ilość próbek (nagrań) & Długość nagrań {[}h{]} \\ \hline
Curved Railway track   & Broken & 1280                  & 3.91                  \\ \hline
Curved Railway track   & Disjointed   & 1280                  & 3.91                   \\ \hline
Curved Railway track   & Normal   & 23264                  & 264.80                 \\ \hline
Curved Railway track   & Obstructing stone   & 1280                  & 3.91                 \\ \hline
First carriage   & Chipped wheel axle   & 2080                  & 6.36                 \\ \hline
First carriage   & Normal   & 25024                  & 270.18                 \\ \hline
Last carriage   & Chipped wheel axle   & 2080                  & 6.36                 \\ \hline
Last carriage   & Normal   & 25024                  & 270.18                 \\ \hline
Straight railway track   & Broken   & 1040                  & 3.18                 \\ \hline
Straight railway track   & Disjointed   & 1040                  & 3.18                 \\ \hline
Straight railway track   & Normal   & 23984                  & 267.00                 \\ \hline
Straight railway track   & Obstructing stone   & 1040                  & 3.18                 \\ \hline
\end{tabular}
\caption{Tabela nagrań dla zbioru ToyTrain}
\end{table}

\FloatBarrier

Na podstawie przeprowadzonej analizy struktury zbiorów danych ToyCar, ToyConveyor oraz ToyTrain można zauważyć wyraźną dominację próbek oznaczonych jako stan „Normal” względem próbek reprezentujących stany uszkodzeń. W każdym z analizowanych zbiorów liczba oraz łączny czas trwania nagrań dla klasy normalnej jest wielokrotnie większy niż dla klas anomalii.

Taka struktura danych wskazuje na silne niezbalansowanie klas, co jest typową cechą danych pochodzących z rzeczywistych systemów przemysłowych, gdzie stany awaryjne występują znacznie rzadziej niż poprawna praca urządzeń. Niezbalansowanie to może prowadzić do sytuacji, w której model uczony bez odpowiednich mechanizmów kompensujących będzie faworyzował klasę dominującą, osiągając pozornie wysoką dokładność kosztem słabej detekcji stanów uszkodzeń.

W kolejnych etapach badań wpływ niezbalansowania klas zostanie oceniony przy użyciu miar takich jak precision, recall oraz F1-score, które lepiej niż dokładność (accuracy) odzwierciedlają jakość klasyfikacji w przypadku danych niezbalansowanych. Uzyskane wyniki pozwolą określić, czy konieczne będzie zastosowanie dodatkowych technik radzenia sobie z niezbalansowaniem klas, takich jak ważenie klas, nadpróbkowanie danych anomalii lub modyfikacja funkcji kosztu modelu.