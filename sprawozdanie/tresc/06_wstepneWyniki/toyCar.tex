
\subsection{ToyCar}
Zbiór ToyCar zawiera nagrania zabawkowego samochody. Model klasyfikuje cztery atrybuty: Shaft (wał), Gears (przekładnie), Tires (opony) oraz Voltage (napięcie zasilania).

\subsubsection{Shaft (Wał)}
Klasyfikacja stanu wału osiągnęła niemal idealne wyniki z dokładnością 99\%. Model praktycznie zawsze rozróżnia wał zgięty (Bent) od normalnego stanu pracy.

\begin{figure}[H]
    \centering
    \includegraphics[width=0.95\textwidth]{reports/report_ToyCar_Shaft.png}
    \caption{Raport klasyfikacji atrybutu Shaft dla zbioru ToyCar}
    \label{fig:toycar_shaft}
\end{figure}

\newpage

\subsubsection{Gears (Przekładnie)}
Klasyfikacja stanu przekładni okazała się najtrudniejszym zadaniem w zbiorze ToyCar. Ogólna dokładność wynosi 93\%, jednak metryki dla klas w których występuje defekt są znacznie niższe. Klasa ``Deformed'' osiąga F1-Score 0.56, a ``Melted'' -- 0.58. Macierz konfuzji pokazuje, że klasy te są często mylone ze sobą oraz z klasą ``Normal''.

\begin{figure}[H]
    \centering
    \includegraphics[width=0.7\textwidth]{reports/report_ToyCar_Gears.png}
    \caption{Raport klasyfikacji atrybutu Gears dla zbioru ToyCar}
    \label{fig:toycar_gears}
\end{figure}

\subsubsection{Tires (Opony)}
Klasyfikacja stanu opon osiągnęła bardzo dobre wyniki z dokładnością 99\%. Niewielkie pomyłki występują między klasami ``Coiled (plastic ribbon)'' i ``Coiled (steel ribbon)''.

\begin{figure}[H]
    \centering
    \includegraphics[width=0.7\textwidth]{reports/report_ToyCar_Tires.png}
    \caption{Raport klasyfikacji atrybutu Tires dla zbioru ToyCar}
    \label{fig:toycar_tires}
\end{figure}

\subsubsection{Voltage (Napięcie)}
Klasyfikacja napięcia zasilania osiąga dokładność 96\%. Stan normalny jest rozpoznawany z bardzo wysoką skutecznością (F1-Score 0.98). Klasy ``Over voltage'' i ``Under voltage'' mają niższe metryki -- odpowiednio F1-Score 0.83 i 0.79. Zauważalna jest tendencja do klasyfikowania przypadków ``Under voltage'' jako ``Normal'', ale nie w drugą stronę.

\begin{figure}[H]
    \centering
    \includegraphics[width=0.95\textwidth]{reports/report_ToyCar_Voltage.png}
    \caption{Raport klasyfikacji atrybutu Voltage dla zbioru ToyCar}
    \label{fig:toycar_voltage}
\end{figure}