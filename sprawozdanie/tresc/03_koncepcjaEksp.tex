\section{Koncepcja części eksperymentalnej}
\subsection{Dane wejściowe}
Danymi wejściowymi w projekcie są krótkie nagrania dźwiękowe o określonej długości, przedstawiające pracę wybranego urządzenia (np. zabawkowego samochodu, przenośnika taśmowego czy miniaturowego pociągu). Każde nagranie pochodzi z zestawu danych, który obejmuje zarówno:
\begin{itemize}
    \item nagrania przedstawiające normalną pracę urządzenia,
    \item nagrania zarejestrowane w sytuacji wystąpienia usterki.
\end{itemize}
Każda z usterek powiązana jest z konkretnymi parametrami pracy urządzenia, takimi jak napięcie zasilania, stan elementów mechanicznych czy stopień zużycia podzespołów.
Przed wykorzystaniem w modelu pliki audio zostaną poddane wstępnemu przetwarzaniu. Obejmuje ono m.in. zastosowanie transformacji Fouriera, która pozwala przekształcić sygnał czasowy w spektrogram.

\subsection{Analizowane wyniki}
Dane wyjściowe modelu to predykcja stanu urządzenia oparta na analizie dźwięku. Predykcja ma charakter:
\begin{itemize}
    \item wieloklasowy – model wybiera jedną z kilku możliwych klas, np. „normalny”, „usterka typu A”, „usterka typu B” itd.,
    \item wieloparametrowy – model w tym samym czasie ocenia wiele aspektów działania urządzenia.
\end{itemize}
Każdy model wykonuje klasyfikację wartości parametrów takich jak:
\begin{itemize}
    \item stan mechaniczny (np. wał, koła, taśma, napęd),
    \item obecność obcych obiektów (np. metallic object),
    \item stan napięcia (under-voltage, normal, over-voltage),
    \item inne cechy charakterystyczne dla konkretnego urządzenia.
\end{itemize}
Oznacza to, że model generuje wiele predykcji równolegle, a każda z nich odnosi się do innego aspektu działania maszyny.

\subsection{Ocena poprawności działania modelu}
Ponieważ system jednocześnie ocenia wiele parametrów, do weryfikacji jego skuteczności konieczne jest użycie zestawu różnych kryteriów.
\begin{enumerate}
    \item Macierz konfuzji (Confusion Matrix) \newline
    Dla każdego parametru generowana jest osobna macierz konfuzji, która pozwala ocenić:
    \begin{itemize}
        \item ile klasyfikacji było poprawnych,
        \item jakie błędy popełnił model (pomyłki typu FP i FN),
        \item które klasy okazały się najtrudniejsze do rozpoznania,
        \item czy model ma tendencję do dominującej klasy („bias”).
    \end{itemize}
    
    \item Precision, Recall, F-measure \newline
    Dla każdej cechy urządzenia wyliczony zostaje zestaw metryk:
    \begin{itemize}
        \item Precision – jak często model się nie myli, kiedy zgłasza daną usterkę,
        \item Recall – jak wiele faktycznie występujących usterek wykrywa,
        \item F-measure – miara łącząca oba poprzednie wskaźniki, pokazująca zależność między precyzją, a skutecznością wykryć.
    \end{itemize}

\end{enumerate}