\subsection{Anomalous Sound Detection Based on Machine Activity Detection\cite{nishida2022anomaloussounddetectionbased}}

Autorzy proponują nowatorskie podejście do nienadzorowanej detekcji anomalii (UASD), które wykorzystuje zadanie pomocnicze polegające na wykrywaniu momentów aktywności maszyny (active/inactive). Metoda ta została opracowana, aby rozwiązać problem spadku skuteczności systemów monitorowania w warunkach przemysłowych, gdzie szum otoczenia jest często zbliżony charakterystyką do dźwięku samej maszyny.

Zasada działania opiera się na treningu modelu klasyfikującego ramki czasowe jako "aktywne" lub "nieaktywne" na podstawie danych normalnych. Rozróżniono dwa scenariusze wnioskowania:
\begin{itemize}
    \item \textbf{UASD-SAD (z etykietami):} Gdy etykiety aktywności są dostępne podczas testów, jako miarę anomalii wykorzystuje się błąd detekcji aktywności (funkcja straty entropii krzyżowej). Anomalne dźwięki powodują błędną klasyfikację stanu maszyny, generując wysoki wynik błędu.
    \item \textbf{UASD-OD-SAD (bez etykiet):} Gdy etykiety nie są dostępne, model służy jako ekstraktor cech. Wektory osadzeń (embeddings) z modelu detekcji aktywności są przekazywane do detektora wartości odstających (np. GMM), który wyznacza wynik anomalii.
\end{itemize}

Eksperymenty przeprowadzone na zbiorze MIMII DUE (Slide rail) wykazały, że proponowane podejście przewyższa konwencjonalne metody oparte na autoenkoderach, szczególnie w warunkach niskiego stosunku sygnału do szumu (SNR) oraz wysokiego podobieństwa szumu do sygnału użytecznego. Autorzy wykazali również, że metoda ta działa komplementarnie do rozwiązań standardowych, a najlepsze rezultaty osiąga się poprzez ich połączenie (ensemble).
