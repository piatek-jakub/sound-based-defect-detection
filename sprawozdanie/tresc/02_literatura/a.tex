\subsection{Detection of defective embedded bearings by sound analysis: a machine learning approach\cite{Bearings}}

Artykuł opisuje rozwiązanie z zakresu uczenia maszynowego (machine learning) dla automatycznej detekcji wadliwych wbudowanych łożysk w urządzeniach AGD na podstawie analizy akustycznej. Jest to podejście konieczne, ponieważ łożyska są instalowane głęboko w urządzeniu na wczesnym etapie produkcji i stają się fizycznie niedostępne po pełnym montażu. Dodatkowo, tradycyjne metody wibracyjne wymagają bezpośredniego dostępu do łożysk, a konwencjonalna analiza Fouriera jest nieskuteczna ze względu na wysoki poziom szumu w sygnałach z linii produkcyjnej.

Kontrola jakości odbywa się poprzez włączenie zmontowanego urządzenia i rejestrowanie sygnału akustycznego za pomocą zewnętrznego czujnika.

Zadanie detekcji potraktowano jako problem klasyfikacji binarnej ("wadliwe" lub "niewadliwe"). Po przeprowadzeniu akwizycji danych (w warunkach laboratoryjnych i na linii produkcyjnej) oraz ich wstępnym przetworzeniu (redukcja szumu, selekcja cech do dziewięciu kluczowych częstotliwości ), przetestowano 67 algorytmów klasyfikacji.

Dla czystych sygnałów laboratoryjnych wybrano Drzewo Decyzyjne ze względu na zadowalającą dokładność w połączeniu z prostotą i intuicyjną implementacją, co było preferowane przez firmę.

Dla głośniejszych sygnałów z linii produkcyjnej najlepszą wydajność osiągnął algorytm IB1 oparty na metryce odległości i metodzie najbliższego sąsiada, który poprawnie klasyfikował wszystkie próbki z tego środowiska.

Rozwiązanie to ma służyć jako narzędzie wspomagające dla techników i zastąpić subiektywną, czasochłonną, ręczną analizę sygnałów akustycznych prowadzoną przez ekspertów.

Wyzwania techniczne przedstawione w artykule:
\begin{itemize}
    \item Ograniczona liczba próbek (dostępne są tylko 34 próbki z linii produkcyjnej),
    \item Wysoki poziom szumu w pomiarach, szczególnie na linii produkcyjnej, wynikający z pracy całego urządzenia oraz innych komponentów. 
\end{itemize}
Autorzy wykazali, że stosunek sygnału do szumu (SNR) jest wystarczająco wysoki, aby maskować ważne składowe, co uniemożliwia skuteczne zastosowanie tradycyjnych technik dekompozycji Fouriera.

Mechanizm proponowanego rozwiązania: 
\begin{itemize}
    \item Akwizycja danych - pomiary w izolowanym laboratorium oraz bezpośrednio na linii produkcyjnej. Każdy sygnał akustyczny reprezentował ciśnienie akustyczne w decybelach dla 24 częstotliwości,
    \item Przetwarzanie wstępne danych - zastosowano selekcję cech, redukując wymiarowość z 24 do dziewięciu najbardziej dyskryminujących częstotliwości (na podstawie różnicy średnich wartości). Wykorzystano również selekcję instancji za pomocą współczynnika Silhouette, aby usunąć potencjalnie błędnie oznaczone (zgodnie z manualną analizą ekspertów) lub redundantne sygnały,
    \item Ewaluacja klasyfikatorów - przeprowadzono porównawcze badanie 67 algorytmów klasyfikacji dla obu środowisk pomiarowych.
\end{itemize}

Wnioski wynikające z publikacji przedstawiają się następująco:
\begin{itemize}
    \item Dla sygnałów laboratoryjnych, pomimo osiągnięcia najwyższej dokładności przez inne metody, jako najlepsze rozwiązanie wybrano Drzewo Decyzyjne (Decision Tree) ze względu na prostotę implementacji i łatwą interpretację reguł klasyfikacji, co było kluczowym wymogiem dla firmy.
    \item Zaproponowana metodyka dostarcza skutecznego narzędzia do automatyzacji kontroli jakości w przemyśle produkcyjnym, przewyższając manualną analizę ekspertów (wykrywając wady, których nie byli w stanie zidentyfikować ludzie).
\end{itemize}