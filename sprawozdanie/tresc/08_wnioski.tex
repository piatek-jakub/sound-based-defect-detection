\section{Wnioski}
Celem niniejszej pracy było zbadanie skuteczności metod analizy sygnałów akustycznych w zadaniu wykrywania anomalii w systemach mechanicznych z wykorzystaniem klasycznych algorytmów uczenia maszynowego. Szczególną uwagę poświęcono metodzie ekstrakcji cech opartych na współczynnikach MFCC oraz wpływowi niezbalansowania klas na jakość klasyfikacji.

Przeprowadzone eksperymenty potwierdziły, że współczynniki MFCC stanowią efektywną i uniwersalną reprezentację cech akustycznych, umożliwiającą rozróżnianie stanów normalnych i anormalnych w analizowanych systemach. Jednakże wyniki uzyskane w początkowej fazie badań, oparte na ograniczonej liczbie współczynników MFCC oraz niewielkim oknie transformaty FFT, wskazywały na niewystarczającą zdolność modelu do wykrywania rzadkich i subtelnych anomalii.

Zwiększenie liczby współczynników MFCC z 20 do 40 oraz zastosowanie większego okna FFT pozwoliło na dokładniejsze odwzorowanie charakterystyki częstotliwościowej sygnału. Dodatkowo rozszerzenie przestrzeni cech o miary spektralne, takie jak \textit{spectral centroid}, \textit{spectral rolloff}, \textit{zero crossing rate}, cechy chromatyczne oraz reprezentację harmoniczną \textit{Tonnetz}, znacząco poprawiło separowalność klas. W efekcie uzyskano istotny wzrost wartości miar \textit{recall} oraz \textit{F1-score} dla klas mniejszościowych, bez pogorszenia jakości klasyfikacji stanów normalnych.

Analiza różnych wariantów modeli wykazała, że wybór strategii klasyfikacji powinien być ściśle uzależniony od docelowego zastosowania systemu diagnostycznego. Modele wieloklasowe umożliwiają identyfikację konkretnego rodzaju anomalii, co jest istotne w kontekście diagnostyki przyczynowej, jednak są bardziej wrażliwe na niezbalansowanie danych. Z kolei modele binarne, w których wszystkie anomalie zostały scalone w jedną klasę \textit{Defect}, charakteryzują się wyższą skutecznością wykrywania uszkodzeń i większą stabilnością, kosztem utraty informacji o typie defektu.

Najwyższą skuteczność osiągnięto w przypadku globalnego modelu binarnego, którego zadaniem była jedynie detekcja obecności anomalii w całym systemie. Uzyskane niemal idealne wyniki potwierdzają, że przy odpowiednio dużym zbiorze danych oraz dobrze dobranej reprezentacji cech, analiza akustyczna może stanowić bardzo skuteczne narzędzie w systemach wczesnego ostrzegania i monitoringu stanu technicznego.

Podsumowując, metoda MFCC, szczególnie w połączeniu z rozszerzonym zestawem cech spektralnych oraz odpowiednimi technikami kompensacji niezbalansowania klas, okazała się skutecznym rozwiązaniem w zadaniach detekcji anomalii. Jednocześnie wyniki pracy wskazują, że kluczowe znaczenie dla jakości klasyfikacji ma nie tylko wybór algorytmu uczenia, lecz przede wszystkim właściwa konfiguracja parametrów ekstrakcji cech oraz świadomy dobór architektury modelu do konkretnego scenariusza zastosowania.