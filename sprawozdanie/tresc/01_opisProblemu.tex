\section{Opis problemu}
Współczesne systemy diagnostyczne coraz częściej wykorzystują analizę dźwięku jako jedno z kluczowych narzędzi do oceny stanu technicznego urządzeń. Metody akustyczne pozwalają na wykrywanie anomalii oraz wczesne diagnozowanie uszkodzeń bez konieczności ingerencji w strukturę maszyny. Analiza sygnałów dźwiękowych jest szczególnie istotna w przypadku urządzeń mechanicznych o złożonej dynamice pracy, gdzie nawet niewielkie odchylenia od normy mogą świadczyć o powstawaniu usterek.

Projekt „Detekcja uszkodzeń na bazie dźwięków” koncentruje się na opracowaniu systemu rozpoznawania cech obiektów oraz identyfikacji ich potencjalnych uszkodzeń na podstawie emitowanych dźwięków. Dane wejściowe pochodzą z publicznie dostępnego ToyADMOS Dataset, obejmującego nagrania dźwięków różnych zabawek mechanicznych, takich jak toy car, toy train, toy conveyor, toy plane oraz innych urządzeń wykonujących stałe, powtarzalne ruchy.

Zbiór danych zawiera zarówno nagrania przedstawiające pracujące poprawnie obiekty, jak i takie, które zawierają celowo wprowadzone uszkodzenia lub anomalie pracy. Dzięki temu możliwe jest zbudowanie i przetrenowanie modeli uczenia maszynowego zdolnych do: 
\begin{itemize}
    \item klasyfikacji rodzaju obiektu na podstawie jego charakterystyk akustycznych,
    \item wykrywania odstępstw od normalnej pracy,
    \item rozpoznawania typu uszkodzenia lub anomalii w strukturze sygnału audio.
\end{itemize}

Głównym celem projektu jest opracowanie systemu zdolnego do automatycznego rozpoznawania różnych cech obiektów z datasetu ToyADMOS na podstawie ich dźwięków, a w szczególności odróżnianie pracy poprawnej od pracy anormalnej. Aby osiągnąć ten cel, konieczne jest przetworzenie sygnałów audio, ekstrakcja cech (np. MFCC, spektrogramy, cechy czasowo-częstotliwościowe), a następnie budowa i trening odpowiednich modeli analitycznych.

Problem, który staramy się rozwiązać, polega na tym, że różne obiekty mogą generować dźwięki o bardzo podobnym charakterze, szczególnie przy niskim poziomie SNR oraz w warunkach naturalnych. Co więcej, anomalie nie zawsze są jednoznacznie słyszalne, a ich identyfikacja wymaga zaawansowanej analizy widma i dynamiki sygnału. Opracowany system powinien więc potrafić wychwycić subtelne różnice w strukturze dźwięku, które nie są łatwe do wykrycia metodami tradycyjnego odsłuchu.

Rozwiązanie tego problemu ma znaczenie zarówno edukacyjne, jak i praktyczne. Z punktu widzenia przemysłu systemy diagnostyki akustycznej znajdują zastosowanie w predykcyjnym utrzymaniu ruchu, automatycznej kontroli jakości, a także w robotyce i IoT. Projekt pozwala również na zapoznanie się z technikami uczenia maszynowego stosowanymi w detekcji anomalii oraz klasyfikacji sygnałów dźwiękowych, z wykorzystaniem rzeczywistego i dobrze przygotowanego zestawu danych.

\begin{figure}[H]
    \centering
    \includegraphics[width=0.8\textwidth]{screeny/x1.png}
    \caption{Fizyczna realizacja obiektów, na bazie których utworzono zbiór ToyADMOS}
\end{figure}